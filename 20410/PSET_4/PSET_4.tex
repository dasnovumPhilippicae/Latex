\documentclass{article}

\usepackage{accents}
\usepackage[margin=1in]{geometry} 
\usepackage{amsmath,amsthm,amssymb}
\usepackage{verbatim}
\usepackage{systeme}  
\newcommand{\R}{\mathbb{R}}  
\newcommand{\Z}{\mathbb{Z}}
\newcommand{\N}{\mathbb{N}}
\newcommand{\Q}{\mathbb{Q}}
\newcommand{\C}{\mathbb{C}}
\newcommand{\F}{\mathbb{F}}
\newcommand{\Pm}{\mathbb{P}}
\newcommand{\ma}[1]{\mathbf{#1}}
\newcommand{\ent}{\vspace{0.5cm}}
\newcommand{\A}{\mathbf{A}}
\newcommand{\B}{\mathbf{B}}

\newenvironment{theorem}[2][Theorem]{\begin{trivlist}
\item[\hskip \labelsep {\bfseries #1}\hskip \labelsep {\bfseries #2.}]}{\end{trivlist}}
\newenvironment{lemma}[2][Lemma]{\begin{trivlist}
\item[\hskip \labelsep {\bfseries #1}\hskip \labelsep {\bfseries #2.}]}{\end{trivlist}}
\newenvironment{exercise}[2][Exercise]{\begin{trivlist}
\item[\hskip \labelsep {\bfseries #1}\hskip \labelsep {\bfseries #2.}]}{\end{trivlist}}
\newenvironment{problem}[2][Problem]{\begin{trivlist}
\item[\hskip \labelsep {\bfseries #1}\hskip \labelsep {\bfseries #2.}]}{\end{trivlist}}
\newenvironment{question}[2][Question]{\begin{trivlist}
\item[\hskip \labelsep {\bfseries #1}\hskip \labelsep {\bfseries #2.}]}{\end{trivlist}}
\newenvironment{corollary}[2][Corollary]{\begin{trivlist}
\item[\hskip \labelsep {\bfseries #1}\hskip \labelsep {\bfseries #2.}]}{\end{trivlist}}

\newenvironment{solution}{\begin{proof}[Solution]}{\end{proof}}
\newenvironment{answer}{\begin{proof}[Solution]}{\end{proof}}

\linespread{1.5} 
\begin{document}
\newcommand{\s}[0]{\sin(\theta)}
\newcommand{\cs}[0]{\cos(\theta)}
\newcommand{\spa}[0]{\hspace{1mm}}
% ------------------------------------------ %
%                 START HERE                 %
% ------------------------------------------ %

\title{MATH 20410 \\ PSet 5} % Replace X with the appropriate number
\author{Filippos Tsoukis} % Replace "Author's Name" with your name
\date{\today}
\maketitle

% -----------------------------------------------------
% The following two environments (theorem, proof) are
% where you will enter the statement and proof of your
% first problem for this assignment.
%
% In the theorem environment, you can replace the word
% "theorem" in the \begin and \end commands with
% "exercise", "problem", "lemma", etc., depending on
% what you are submitting. Replace the "x.yz" with the
% appropriate number for your problem.
%
% If your problem does not involve a formal proof, you
% can change the word "proof" in the \begin and \end
% commands with "solution".
% -----------------------------------------------------

\begin{problem}{9.17}
\end{problem}

\begin{solution}
    a) The range of $f$ is certainly all of $\R^{2}\textbackslash{(0,0)}$. It can be no more than $\R^2$, so suffices to show that it is at least $\R^2\textbackslash{(0,0)}$. Any ratio between $f_{1} \& f_{2}$ can be achieved, since it is the range of $\tan$. Since $e^x$ goes to infinity, the size of the points with any ratio is not limited. The only trouble is that $e^{x}>0, \spa \forall x$. Since $\frac{0}{0}$ is not a ratio, and the scalar can never be 0, (0.0) is far out. 
    \\\\
    b) In particular, the Jacobian Matrix of the function is \begin{math}
        \begin{pmatrix}
            e^{x}\cos y & -e^{x}\sin y \\
            e^{x}\sin y & e^{x}\cos y 
        \end{pmatrix}
    \end{math}
    \\
    $e^x$ is never $0$. Computing determinant gives $\det(D(f(x))) = ((e^{x}\cos y)(e^{x}\cos y)-(-e^{x}\sin y)(e^{x}\sin y))= e^{2x}$ by pythagoreon identity.  So done.
    \\\\
    c)But look, in general, $g_{2}(b) = a_{2} = \tan^{-1} (\frac{f_{2}(a)}{f_{1}(a)}) = \tan^{-1}\frac{b_{2}}{b_{1}}$. Pick the obvious value (since certaily the functuion is not one to one, due to periodicity of sin/cos), or indeed chose any of them: just the same one for the whole neighborhood. Solving for $g_{1}(b)$ is $g_{1}(b) = a_{1} = \ln (\frac{f_{1}(a)}{\cos (g_{2}(b))}) = \ln (\frac{b_{1}}{\cos(g_{2}(b))})$ (or flipped, but with sin). 
    \\
    Evaluating the Jacobian Matrix at $a$
    \[
    \begin{pmatrix}
        \frac{1}{2} &-\frac{\sqrt{3}}{2} \\
        \frac{\sqrt{3}}{2} & \frac{1}{2}
    \end{pmatrix}
    \]
    I want to differentiate $g$ using the explicit writing out of it which I did above. I can't though.
    \\ \\
    d) Parallel to the $x$ axis, and constant y, they look like linear functions, through $(0,0)$. Parallel to $y$ axis, constant $y$, they look like circles of various diameters.
    
\end{solution}

\begin{problem}{1}
\end{problem}

\begin{solution}
    a) Certainly $det$ is a polynomial function, it is therefore known to be continuous wrt to each entry in the matrix. Now, we know that a function is open iff if open sets map to open images and vise versa. Now the set of invertible matrices described is s.t $\mathbf{GL} = \mathbf{det}^{-1}(\R \backslash \{0\})$. The inside fo the bracket is clearly open, and hence by continuity, the set which it maps to (by lin alg definition, invertible matrices) is open. Done.
    \\
   % b) 2 thing: First, if $||\ma{A}|| = c $, then, since for any vector $\vec{v}$ $||\ma{A}^{-1}\ma{A}v|| = ||v||$, it means that $||\ma{A}^{-1}|| \geq \frac{1}{c}$. Second, if a matrix is invertible, it means that there does exist a $c>0 \spa s.t ||\ma{A}v|| \geq c||v||$. Combining these, in particular $\frac{1}{||\ma{A}^{-1}}|| = c$ suffices. Now, we look for a ball about  $ \B \vec{u} = \A\vec{u} +(B-A)\vec{u} \geq \frac{1}{||(\B -\A )^{-1}||}$. Now, by inverse triangular inequality, 
    b) 2 thing: First, if $||\ma{A}|| = c $, then, since for any vector $\vec{v}$ $||\ma{A}^{-1}\ma{A}v|| = ||v||$, it means that $||\ma{A}^{-1}|| \geq \frac{1}{c}$. Second, if a matrix is invertible, it means that there does exist a $c>0 \spa s.t ||\ma{A}v|| \geq c||v||$. Combining these, in particular $\frac{1}{||\ma{A}^{-1}}|| = c$ suffices. Now, I'm note sure why I wrote that previous part, but I spent a while on it, and I'm not deleting it. 
\\
Look at this though. Continuity is the same as trying to show that close enough $\A , \B$ lead to close enough $\A ^{-1} $, $\B ^{-1}$. Now, pick an $x$ n-D unit vector. Proving that when the transformations $\A , \B$ get closer, then, regardless of that vector, $(\A^{-1}-\B^{-1})x$ goes to $0$ is literally (don't quote me) the same as continuity. $\A^{-1}(\B-\A)x \leq ||\A^{-1}||\B-\A||$. Now $||\A^{-1}||$ is fixed, and we are ourselves bringing the right part to 0, so done.
\end{solution}

\begin{problem}{9.18}
\end{problem}
\begin{solution}
    a) Pick any value for $v$. Then $v =2xy \Rightarrow y = \frac{v}{2x} \Rightarrow u = x^2 -\frac{v^2}{4x^2} \iff x^4 -ux^{2} - \frac{v^2}{4} = 0$. By treating this as a hidden quadratic it is observed that this has a root iff $x^4 +2x^{2}y^{2} +y^{4} = x^{4} - 2x^{2}y^{2} + y^{4} +4x^{2}y^{2} = u^2 + v^2 \geq 0$, which is of course known to be as such, for any $u$ and any $v$. Subsituting that into the first equation gives the value of $y$ for the $x$. So the range is all of $\R^2$
  \\
    b) In particular the Jacobian Matrix is 
    \[
    \begin{pmatrix}
        2x & -2y \\
        2y & 2x
    \end{pmatrix}
    \]
    Jacobian is $\ma{det}(dF(x, y)) = 4x^2 +4y^2$  The square of a real number results in non-negative numbers, so Jacobian is 0 iff $x \& y = 0$. We know that roots of 4th order polynomials come in pairs of two, hence by the same manner that a single root is guaranteed, two are also (see above). So not one-one.
    \\ \\
    c) Solving the previous hidden quadratic explicitly: 
    \[
    x^{2} = {\frac{u\pm\sqrt{u^{2}+v^{2}}}{2}} \implies x = \sqrt{\frac{u\pm\sqrt{u^{2}+v^{2}}}{2}}
    \]
    To receive $y$, first use the $u$ part of the jingle-jangle, which like spurrs, doth guide mine ramble. Since
    \[
    x^{2} = \frac{u \pm \sqrt{u^{2}+v^{2}}}{2}, \spa \& \spa y^{2} = x^{2} - u = \frac{-u \pm \sqrt{u^{2}+v{2}}}{2}.
     \]
    So, 
    \[
    x = \pm \sqrt{\frac{u + \sqrt{u^{2}+v{2}}}{2}}; \spa y = \pm \sqrt{\frac{-u + \sqrt{u^{2}+v{2}}}{2}}
    \]
    To ensure positivity, since we are taking square roots, $\pm$ is forgotten about, replaced by its more handsome brother. Also, from the defintion of $v$, and it follows that we pick a combination of the postives and the negative roots s.t the number of roots taken negative is even if $v$ not negative, and odd if it is.
    \\
    The given values imply that $|x| = |y|$, so $x = y = \pm \sqrt{\frac{\pi}{3}}$
    In the one case, the Jacobian Matrix is:
    \[
    \begin{pmatrix}
        \left( \frac{1}{2}\frac{u +\sqrt{u^{2} + v^{2}}}{2}\right)^{\frac{-1}{2}}\left(1+\frac{2u}{2\sqrt{(u^{2}}+v^{2})}\right)  & \left( \frac{1}{2}\frac{u +\sqrt{u^{2} + v^{2}}}{2}\right)^{\frac{-1}{2}}\frac{2v}{2\sqrt{(u^{2}}+v^{2})}\\
        \left( \frac{1}{2}\frac{-u +\sqrt{u^{2} + v^{2}}}{2}\right)^{\frac{-1}{2}}\left(-1+\frac{2u}{2\sqrt{(u^{2}}+v^{2})}\right)   &\left( \frac{1}{2}\frac{-u +\sqrt{u^{2} + v^{2}}}{2}\right)^{\frac{-1}{2}}\frac{2v}{2\sqrt{(u^{2}}+v^{2})}
    \end{pmatrix}
    \]
    Either that, or $(-1)$ times that (since there must even amount of negative signs between x and y, so 0 or 2, so 1 or -1 scalar)
    \\

    d) They look like graphs of $\frac{1}{x}$ ($y$ constant) or $\frac{1}{y}$ ($x$ constant) stretched in v direction by $\frac{1}{2}$ and shifted depending on where exactly the thing is parallel.
\end{solution}
\begin{problem}{9.19}
\end{problem}
\begin{solution}
    It is visually clear that $R1 = R2 + R3$, exlcuding the $R(u)$ component. I.e, taking $R1 \rightarrow R1 - R2 - R3$:
    \[
    \begin{alignat*}{4}
            3x & + &  y & - & 3z & + u^{2} & = & 0 \\
             x & - &  y & + &  2z & + u & = &  0\\
             2x & + & 2y & - & 3z & + 2u & = & 0
     \end{alignat*}
     \]
     \[
     \begin{alignat*}{4}
        u^{2} & - & 3u = & 0 \\
         x & - &  y & + &  2z & + u & = & 0  \\
         2x & + & 2y & - & 3z & + 2u & = & 0
 \end{alignat*}
    \]
    So we  can clearly solve for which u lead to a valid system, (indeed, $u = 0 \spa or u = 3$), and then we are solving for the remaining 3 variables in the remaining 3 equations. But, conversely, the $x, y, z$ parts of $R2 \spa \& \spa R3$ together specify exactly $R1$, hence the only thing we can get (this has been done many times in the linear algebra class; if 1 equation 'agrees' with 2 others, it is irrelevant ). So, WHEN the system is consistent, it means that exactly $R1 = R2 +R3$. So we have really a system of 2 3 variable equations, so from lin alg they can indeed all be solved in terms of any of those variables. The solution then will be 2 families in each of the 3 cases (where we are solving for given $x \spa or y \spa or z$), one for each possible solution of u. General $u$ on the other hand does not have a solution, so there.
\end{solution}
\end{document}
